\documentclass[10pt]{article}

\usepackage[spanish]{babel}
\usepackage[utf8]{inputenc}
\usepackage[OT1]{fontenc}
\usepackage{amsfonts, amsmath, amsthm, amssymb}
\usepackage{graphicx}
\usepackage{listings}
\usepackage[margin=0.8in]{geometry}
\usepackage{xcolor}

\newcounter{countCode}

\lstnewenvironment{code} [1][caption=Ponme caption, label=default]{%
\renewcommand*{\lstlistingname}{Listado} 
\setcounter{lstlisting}{\value{countCode}}
\lstset{ %
		language=java,
		basicstyle=\ttfamily\footnotesize,      	% the size of the fonts that are used for the code
		numbers=left,                   			% where to put the line-numbers
		numberstyle=\sc,      						% the size of the fonts that are used for the line-numbers
		stepnumber=1,                  				% the step between two line-numbers. 
		numbersep=5pt,                 				% how far the line-numbers are from the code
		numberstyle=\color{red!50!blue},
		backgroundcolor=\color{lightgray!20},
		rulecolor=\color{blue},
		keywordstyle=\color{red}\bfseries,
		showspaces=false,               			% show spaces adding particular underscores
		showstringspaces=false,         			% underline spaces within strings
		showtabs=false,                 			% show tabs within strings adding particular underscores
		frame=single,                   			% adds a frame around the code
		framexleftmargin=0mm,
		numberblanklines=false,
		xleftmargin=5pt,
		breaklines=true,
		breakatwhitespace=true,
		breakautoindent=true,
		captionpos=t,
		texcl=true,
		tabsize=2,                      % sets default tabsize to 3 spaces
		extendedchars=true,
		inputencoding=utf8, 
		escapechar=\%,
		morekeywords={print, println, size, background, strokeWeight, fill, line, rect, ellipse, triangle, arc, save, PI, HALF_PI, QUARTER_PI, TAU, TWO_PI, width, height,},
		emph=[1]{print,println,}, emphstyle=[1]{\color{blue}}, % Mis palabras clave.
		emph=[2]{width,height,}, emphstyle=[2]{\bf\color{violet}}, % Mis palabras clave.
		emph=[3]{PI, HALF_PI, QUARTER_PI, TAU, TWO_PI}, emphstyle=[3]\color{orange!50!violet}, % Mis palabras clave.
		emph=[4]{line, rect, ellipse, triangle, arc,}, emphstyle=[4]\color{green!70!black}, % Mis palabras clave.
		#1}}{\addtocounter{countCode}{1}}



\title{Filtros de Kalman}
\author{José Ramírez-Gómez}
\date{\today}

\begin{document}
	\maketitle
	\tableofcontents 
	\newpage
	
%\begin{abstract}
%	\noindent En este documento se explicara que es un filtro de Kalman, se hará su %deducción y se mostrarán sus aplicaciones. 
%\end{abstract}

\section{Introducción}
	\noindent El filtro de Kalman es un estimador lineal discreto que permite la reconstrucción del estado a partir de mediciones ruidosas. Aunque existían otros filtros previos, el filtro de Kalman aporta una unificación de los métodos que existían previamente, un algoritmo que no se basa en la transformada de Fourier y que además es estable, robusto y confiable.\\
	
	\noindent El filtro es un estimador de tipo predictor-corrector que usa información de diversas fuentes para reducir la incertidumbre de las mediciones y lograr una mejor estimación del estado:
		
		\begin{itemize}
			\item Conocimiento del sistema.
			\item Dinámica de los dispositivos de medición.
			\item Descripción estadística de los ruidos del sistema.
			\item Información de las condiciones iniciales.
		\end{itemize}

\section{Deducción del Filtro de Kalman}
	\noindent Para la deducción del filtro de Kalman se deben hacer algunas suposiciones:
		
		\begin{enumerate}
			\item El modelo es lineal.
			\item El modelo es observable.
			\item El ruido blanco de las mediciones $\mathbf{v}$ y en las variables de estado $\mathbf{w}$ es gaussiano.
		\end{enumerate}
	
	\noindent Se parte del siguiente modelo:
	
		\begin{equation*}
			\begin{cases}
				\mathbf{x}(k+1) = \mathbf{\Phi x}(k) + \mathbf{\Gamma u}(k) + \mathbf{G w}(k)\\
				\mathbf{y}(k) = \mathbf{C x}(k) + \mathbf{v}(k)
			\end{cases}
		\end{equation*}
	
		\begin{equation*}
			p(\mathbf{v}) \sim N(0,\mathbf{R}) \hspace{1cm} p(\mathbf{w}) \sim 	N(0,\mathbf{Q}) \hspace{1cm} E\mathbf{v}(k)\mathbf{w}^{T}(k) = 0
		\end{equation*}
	
	\noindent Donde $\mathbf{P, Q, R}$ son la matriz de covarianzas del estado, la matriz de covarianzas del ruido del proceso y la matriz de covarianzas del ruido de medición. Estas matrices se pueden calcular así:
	
		\begin{equation*}
			\begin{cases}
				\mathbf{P}(k) = E[\mathbf{x}(k) - \hat{\mathbf{x}}(k)][\mathbf{x}(k) - \hat{\mathbf{x}}(k)]^{T}\\
				\mathbf{Q}(k) = E\mathbf{w}(k)\mathbf{w}^{T}(k) \\
				\mathbf{R} = E\mathbf{v}(k)\mathbf{v}^{T}(k)
			\end{cases}
		\end{equation*}	
	
	\noindent La ecuación de predicción del estado $\hat{\mathbf{x}}(k+1|k)$ a partir de la ecuación del estado actual $\hat{\mathbf{x}}(k|k)$ es:
	
		\begin{equation*}
			\hat{\mathbf{x}}(k+1|k) = \mathbf{\Phi\hat{x}}(k|k) + \mathbf{\Gamma u}(k)
		\end{equation*}
	
	\noindent El error del estado predicho es:
	
		\begin{align*}
			\hat{\mathbf{e}}(k+1|k) 	&=	\mathbf{x}(k+1) - \hat{\mathbf{x}}(k+1|k)\\
										&=	[\mathbf{\Phi x}(k) + \mathbf{\Gamma u}(k) + \mathbf{G w}(k)] - [\mathbf{\Phi}\hat{\mathbf{x}}(k) + \mathbf{\Gamma u}(k)] \\
										&=	\mathbf{\Phi\hat{e}}(k|k) + \mathbf{G w}(k)
		\end{align*}
\end{document}