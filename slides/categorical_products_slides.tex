\documentclass[11pt, aspectratio = 169,usenames,dvipsnames]{beamer}

%----- beamer configuration -----%
\usecolortheme{default}
\useinnertheme{circles}
\setbeamertemplate{navigation symbols}{}
\setbeamertemplate{footline}[page number]{}
\setbeamersize{text margin left=5mm,text margin right=5mm} 
\definecolor{sapgreen}{rgb}{0.31, 0.49, 0.16}
\setbeamercolor{math text}{fg=sapgreen}

%----- packages -----%
\usepackage{graphicx}
\usepackage{xcolor}
\usepackage{hyperref}
\usepackage{tikz}
\usepackage{braket}
\usepackage{setspace}

%----- references -----%
\hypersetup{
	colorlinks=true,
	citecolor=blue,
	linkcolor=blue,
	filecolor=magenta,      
	urlcolor=cyan,
	pdftitle={Overleaf Example},
	pdfpagemode=FullScreen,
}

%----- comands -----%
\newcommand{\coldef}[1]{\textcolor{Fuchsia}{#1}}
\newcommand{\ca}[1]{\mathcal{#1}}

%----- titlepage -----%
\title{Products In Categories}

\author{José Ramírez-Gómez}

\institute{Universidad EAFIT}

\date{2023-2}

%--------------------------------------------------
\begin{document}
	\onehalfspacing
	
	\frame[plain]{\titlepage}
	
	\begin{frame}[t]
		\frametitle{Groups and Abelian Groups}
		\begin{block}{Definition: Group}
			\onslide<1->{A group is a set $G$ together whit an operation $*$ on $G$ such that each of the following axioms is satisfied:}
			\begin{itemize} 
				\onslide<2->{\item\textbf{Associativity:} $a * (b * c) = (a * b) * c$ for all $a,b,c \in G$.}
				\onslide<3->{\item\textbf{Existence of identity:} There is an element $e \in G$ such that $a * e = e * a = a$ for each $a \in G$.}
				\onslide<4->{\item\textbf{Existence of inverses:} For each $a \in G$ there is an element $b \in G$ such that $a * b = b * a = e$.}
			\end{itemize}
		\end{block}
		
		\onslide<5->{
			\begin{block}{Definition: Abelian Group}
				A group $\mathcal{G}$ is said to be Abelian if the group operation is commutative $a * b = b * a$.
			\end{block}
					}
	\end{frame}
	
	\begin{frame}[t]
		\frametitle{Rings and Fields}
		\begin{block}{Definition: Ring}
			\onslide<1->{A ring is a set $R$ together with two operations on $R$ called addition $(a+b)$ and multiplication $(ab)$, such that:}
			\begin{itemize}
				\onslide<2->{\item $R$ with multiplication is an Abelian group.}
				\onslide<3->{\item Multiplication is associative.}
				\onslide<4->{\item $a(b+c) = ab + ac$ and $(a+b)c = ac + bc$ for all $a, b,c \in R$.}
			\end{itemize} 
		\end{block}
		
		\onslide<5->{\begin{block}{Definition: Field}
				A commutative ring in which the set of nonzero elements forms a group with respect to multiplication is a field \cite{durbin2008modern}.
		\end{block}}
	\end{frame}
	
	\begin{frame}[t]
		\frametitle{Fields and Products}
		\onslide <1->{\begin{block}{Case 1: Different Characteristic}
				Let $K$ and $L$ be two fields with different characteristics. Then $K\times L$ if it exists, must be a field which can map to $K$ and $L$. Precisely this means that $K \times L$ has the same characteristic as $K$ and $L$ which is impossible since the characteristic of a field is unique and by hypothesis the characteristics of $K$ and $L$ are different.
		\end{block}}
	\end{frame}
	
	\begin{frame}[t]
		\frametitle{Fields and Products}
		\begin{block}{Case 2: Same Characteristic}
			\begin{itemize}
				\onslide<1->{\item In the category of fields of some fixed characteristic, let $K$ be a field whit an endomorphism $e: K \to K$ different to the identity. Suppose that the product $P = K \times K$ exists and let $p: P \to K$, $q: P \to K$ be the projection arrows.}
			
				\onslide<2->{\item Considering $L = K$ and the identity arrows $f = g = 1: L \to K$. By the UMP of products there must exists an arrow $h: L \to P$ such that $ph = f$ and $qh = g$.}
			
				\onslide<3->{\item Since $f$ and $g$ are the identity and every map of fields is injective (mono) then $h$ must be an isomorphism and furthermore $p = q$ is its inverse.}
				
				\onslide<4->{\item Now consider $g' = e : L \to K$, there is then an $h': L \to P$ such that $ph' = f$ and $qh' = g'$. Since $p = q$, this means $f = g'$ which is a contradiction since we assumed $e \neq 1$. \qed}
			\end{itemize}
		\end{block}
	\end{frame}
	
	
	\begin{frame}[t]
		\frametitle{References}
		\bibliographystyle{plain}
		\bibliography{references}
	\end{frame}
	
\end{document}