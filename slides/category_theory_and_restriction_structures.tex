\documentclass[10pt, aspectratio = 43,usenames,dvipsnames]{beamer}

%----- beamer configuration -----%
\usecolortheme{default}
\useinnertheme{circles}
\setbeamertemplate{navigation symbols}{}
\setbeamertemplate{footline}[page number]{}
\setbeamersize{text margin left=5mm,text margin right=5mm} 
\definecolor{sapgreen}{rgb}{0.31, 0.49, 0.16}
\setbeamercolor{math text}{fg=sapgreen}
%\usepackage[spanish]{babel}
\newtheorem{definicion}{Definición}
\newtheorem{ejemplo}{Ejemplo}
\newtheorem{ejemplos}{Ejemplos}

%----- packages -----%
\usepackage{graphicx}
\usepackage{xcolor}
\usepackage{hyperref}
\usepackage{tikz-cd}
\usepackage{braket}
\usepackage{setspace}
\usepackage{quiver}

%----- references -----%
\hypersetup{
	colorlinks=true,
	citecolor=magenta,
	linkcolor=blue,
	filecolor=magenta,      
	urlcolor=cyan,
	pdftitle={Overleaf Example},
	pdfpagemode=FullScreen,
}

%----- comands -----%
\newcommand{\coldef}[1]{\textcolor{Fuchsia}{#1}}
\newcommand{\ca}[1]{\mathcal{#1}}

%----- titlepage -----%
\title{Teoría de Categorías y Estructuras de Restricción}

\author{José Ramírez-Gómez}

\institute{Universiad EAFIT}

\date{Septiembre 2024}

%--------------------------------------------------
\begin{document}
	\onehalfspacing
	
	\frame[plain]{\titlepage}
	
	\section{Introducción}
	\begin{frame}[t]
		\frametitle{¿Qué es la teoría de categorías?}
		La teoría de categorías es una rama de las matemáticas que:
		\begin{itemize}
			\item Estudia conexiones entre diferentes ideas.
			\item Permite entender el orden y las estructuras.
			\item Se centra en las relaciones.
			\item Permite ver bajo qué contexto las cosas son equivalentes.
		\end{itemize}
	\end{frame}
	
	\begin{frame}[t]
		\begin{figure}[h!]
			\frametitle{¿Qué es la teoría de categorías?}
			\begin{center}
				\includegraphics[width= 0.9 \textwidth]{images/meme.jpg}
				\caption{Broma}
			\end{center}
		\end{figure}
	\end{frame}
	
	\begin{frame}[t]
		\frametitle{Diagramas conmutativos}
		\begin{block}{Idea}
			Un diagrama conmutativo es un diagrama en el cual todos los caminos que comiencen y terminen en el mismo lugar determinan el mismo resultado.
		\end{block}
		\begin{block}{Ejemplo}
			\[\begin{tikzcd}[ampersand replacement=\&,cramped]
				{\text{Marinilla}} \&\& {\text{Rionegro}} \\
				\\
				{\text{Guarne}} \&\& {\text{Medellín}}
				\arrow[from=1-1, to=1-3]
				\arrow[from=1-1, to=3-1]
				\arrow[from=1-3, to=3-3]
				\arrow[from=3-1, to=3-3]
			\end{tikzcd}\]
		\end{block}
	\end{frame}
	
	\begin{frame}[t]
		\frametitle{¿Qué es una categoría?}
		Informalmente, una categoría, que denotamos como $\mathcal{C}$, está formada por:
		\begin{itemize}
			\item \textbf{Objetos:} Pueden ser números, formas, estructuras matemáticas, o personas. Usualmente denotamos los objetos con letras mayúsculas $A,B,C,\ldots$.
			\item \textbf{Flechas:} Denotan relación entre los objetos: ``menor que'', ``es madre de'', ``es isomorfo a''. Denotamos las flechas con diagramas $A\overset{f}{\longrightarrow}B$.
			\item \textbf{Identidad:} Cada objeto está relacionado con él mismo. Es decir, para cada objeto $A$ hay una flecha $A\overset{1_{A}}{\longrightarrow}A$.
			\item \textbf{Composición:} Para cada par de flechas $A\overset{f}{\longrightarrow}B\overset{g}{\longrightarrow}C$
			debe existir una flecha $A\overset{h}{\longrightarrow}C$.
		\end{itemize}
	\end{frame}

	\begin{frame}[t]
		\frametitle{¿Qué es una categoría?}
		\begin{itemize}
			\item \textbf{Ley de identidades:} La flecha identidad asociada a cada objeto hace que los siguientes diagramas sean conmutativos
			\[\begin{tikzcd}[ampersand replacement=\&,cramped]
				A \&\& A \&\& A \&\& B \\
				\\
				\&\& B \&\&\&\& B
				\arrow["{1_{A}}", from=1-1, to=1-3]
				\arrow["f"', dashed, from=1-1, to=3-3]
				\arrow["f", from=1-3, to=3-3]
				\arrow["f", from=1-5, to=1-7]
				\arrow["f"', dashed, from=1-5, to=3-7]
				\arrow["{1_{B}}", from=1-7, to=3-7]
			\end{tikzcd}\]
			\item \textbf{Ley asociativa:} Cada terna de flechas con la forma $A\overset{f}{\longrightarrow}B\overset{g}{\longrightarrow}C\overset{h}{\longrightarrow}D$ debe hacer que el siguiente diagrama sea conmutativo
			\[\begin{tikzcd}[ampersand replacement=\&,cramped]
				A \&\& B \\
				\\
				\&\& C \&\& D
				\arrow["f", from=1-1, to=1-3]
				\arrow["{g\circ f}"', from=1-1, to=3-3]
				\arrow["g", from=1-3, to=3-3]
				\arrow["{h\circ g}", from=1-3, to=3-5]
				\arrow["h", from=3-3, to=3-5]
			\end{tikzcd}\]
		\end{itemize}
	\end{frame}

	\begin{frame}[t]
		\frametitle{Ejemplo: Números naturales}
		Partiendo de los números naturales, podemos construir una categoría de la siguiente forma
		\begin{itemize}
			\item \textbf{Objetos:} Números naturales $0,1,2,3,\ldots$.
			\item \textbf{Flechas:} $A\longrightarrow B$ siempre y cuando $A\leq B$.
			\item \textbf{Identidad:} Todo número natural es menor o igual que él mismo.
		\end{itemize}
		\begin{block}{¿Cómo se vería esta categoría?}
			\[\begin{tikzcd}[ampersand replacement=\&,cramped]
				0 \&\& 1 \&\& 2 \&\& 3 \&\& \ldots
				\arrow[from=1-1, to=1-1, loop, in=55, out=125, distance=10mm]
				\arrow[from=1-1, to=1-3]
				\arrow[curve={height=6pt}, from=1-1, to=1-3]
				\arrow[curve={height=12pt}, from=1-1, to=1-5]
				\arrow[curve={height=18pt}, from=1-1, to=1-7]
				\arrow[curve={height=24pt}, from=1-1, to=1-9]
				\arrow[from=1-3, to=1-3, loop, in=55, out=125, distance=10mm]
				\arrow[from=1-3, to=1-5]
				\arrow[from=1-5, to=1-5, loop, in=55, out=125, distance=10mm]
				\arrow[from=1-5, to=1-7]
				\arrow[from=1-7, to=1-7, loop, in=55, out=125, distance=10mm]
				\arrow[from=1-7, to=1-9]
			\end{tikzcd}\]
		\end{block}
	\end{frame}
	
	\begin{frame}[t]
		\frametitle{Ejemplo: $\mathsf{Set}$}
		$\mathsf{Set}$ es la categoría cuyos objetos son conjuntos y sus flechas son funciones entre conjuntos. En esta categoría
		\begin{itemize}
			\item \textbf{Flecha identidad:} Es la función identidad
			\begin{align*}
				\mathbf{1}_{X}: X&\to X\\
				x&\mapsto x
			\end{align*}
			\item \textbf{Composición:} Es la composición de funciones. Si $f:X\to Y$ y $g: Y\to Z$ son dos funciones, luego
			\begin{align*}
				g\circ f: X&\to Z\\
				x&\mapsto g(f(x))
			\end{align*}
			es su función composición.
		\end{itemize}
	\end{frame}
	
	
	\begin{frame}[t]
		\frametitle{Otros ejemplos}
		\begin{itemize}
			\item $\mathsf{Vect}$: Los objetos son espacios vectoriales y los morfismos son transformaciones lineales.
			\item $\mathsf{Pos}$: Los objetos son conjuntos parcialmente ordenados y los morfismos son funciones monótonas.
			\item $\mathsf{Top}$: Los objetos son espacios topológicos y los morfismos son los mapeos continuos.
			\item $\mathsf{Grp}$: Los objetos son grupos y los morfismos son homomorfismos de grupos.
		\end{itemize}
	\end{frame}

	\begin{frame}[t]
		\frametitle{Restricción de una función}
		\begin{block}{Función de inclusión}
			Sea $A$ un conjunto. Todo subconjunto $S\subseteq A$ define una función de inclusión $S\hookrightarrow A$ que envía cada $s\in S$ en $s\in A$. 
		\end{block}
		\begin{figure}
			\includegraphics[width=0.3 \textwidth]{images/inclusion.jpg}
			\caption{Idea de la función de inclusión}
		\end{figure}
	\end{frame}

	\begin{frame}[t]
		\frametitle{Restricción de una función}
		\begin{block}{Definición}
			Sea $f: X\to Y$ una función del conjunto $X$ en el conjunto $Y$. Si $W$ es un subconjunto de $X$, entonces la restricción de $f$ respecto a $A$ es la función
			\begin{align*}
				f|_{W}: W&\to Y
			\end{align*}
			dada por $f|_{W}(x)\mapsto f(x)$.
		\end{block}
		\begin{figure}
			\includegraphics[width=0.4 \textwidth]{images/rest_conj.jpg}
			\caption{Idea de la restricción de una función}
		\end{figure}
	\end{frame}
	
	\begin{frame}
		\begin{center}
		\LARGE¿Podemos extender esta idea a una categoría?
		\end{center}
	\end{frame}

	\begin{frame}[t]
		\frametitle{Categorías restrictivas}
		\begin{block}{Restricción (categoría)}
			Una restricción en una categoría $\mathcal{C}$ es un asignación de una flecha $\bar{f}: A\to A$ para cada flecha $f:A\to B$ tal que
			\begin{itemize}
				\item Para toda $f$ en $\mathcal{C}$, $f\circ\bar{f} = f$
				\[\begin{tikzcd}[ampersand replacement=\&,cramped]
					A \&\&\& A \\
					\\
					\\
					\&\&\& B
					\arrow["{\bar{f}}", from=1-1, to=1-4]
					\arrow["f"', from=1-1, to=4-4]
					\arrow["f", from=1-4, to=4-4]
				\end{tikzcd}\]
			\end{itemize}
		\end{block}
	\end{frame}

	\begin{frame}[t]
		\frametitle{Categorías restrictivas}
		\begin{itemize}
			\item Para toda $f:A\to B$ y $g:A\to C$, $\bar{f}\circ\bar{g} = \bar{g}\circ\bar{f}$
			\[\begin{tikzcd}[ampersand replacement=\&,cramped]
				A \&\& A \\
				\\
				A \&\& A
				\arrow["{{\bar{f}}}", from=1-1, to=1-3]
				\arrow["{{\bar{g}}}"', from=1-1, to=3-1]
				\arrow["{{\bar{g}}}", from=1-3, to=3-3]
				\arrow["{{\bar{f}}}"', from=3-1, to=3-3]
			\end{tikzcd}\]
			\item Para toda $f:A\to B$ y $g: A\to C$, $\overline{g\circ\bar{f}} = \bar{g}\circ\bar{f}$
			\[\begin{tikzcd}[ampersand replacement=\&,cramped]
				A \&\& A \\
				\\
				\&\& A
				\arrow["{\bar{f}}", from=1-1, to=1-3]
				\arrow["{\overline{g\bar{f}}}"', from=1-1, to=3-3]
				\arrow["{\bar{g}}", from=1-3, to=3-3]
			\end{tikzcd}\]
		\end{itemize}
	\end{frame}
	
	\begin{frame}[t]
		\frametitle{Categorías restrictivas}
		\begin{itemize}
			\item Para toda $f:A\to B$ y $g:B\to C$, $\bar{g}\circ f = f\circ\overline{g\circ f}$
			\[\begin{tikzcd}[ampersand replacement=\&,cramped]
				A \&\&\& B \\
				\\
				\\
				A \&\&\& B
				\arrow["f", from=1-1, to=1-4]
				\arrow["{\overline{g\circ f}}"', from=1-1, to=4-1]
				\arrow["{\bar{g}}", from=1-4, to=4-4]
				\arrow["{f}"', from=4-1, to=4-4]
			\end{tikzcd}\]
		\end{itemize}
		
		\begin{block}{Categoría restrictiva}
			Una categoría restrictiva es una categoría $\mathcal{C}$ junto a una estructura de restricción.
		\end{block}
	\end{frame}

	\begin{frame}[t]
		\frametitle{Categorías restrictivas}
		\begin{block}{Notas}
			\begin{itemize}
				\item Es importante tener en cuenta que la estructura de restricción no es una propiedad de la categoría. Una categoría puede tener más de una estructura de restricción.
				
				\item Toda categoría tiene al menos una estructura de restricción. Si tomamos $\bar{f} = 1$ como le restricción de cada flecha obtenemos la estructura de restricción trivial.
			\end{itemize}
		\end{block}
	\end{frame}
	
	\begin{frame}
		\begin{center}
			\LARGE ¿Para qué sirve todo esto?
		\end{center}
	\end{frame}
	
	\begin{frame}
		\begin{center}
			\huge¡GRACIAS!
		\end{center}
	\end{frame}
	
	\begin{frame}[t]
		\frametitle{¿Qué es una categoría?}
		\begin{definicion}[Categoría]
			Una categoría $\mathcal{C}$ está compuesta de:
			\begin{itemize}
				\item Una colección de objetos, $Obj(\mathcal{C})$, que denotaremos con las letras $A,B,C$, etc.
				\item Una colección de flechas o morfismos, $Mor(\mathcal{C})$, que denotaremos con las letras $f,g,h$, etc.
				\item Dos mapeos, $dom, cod: Mor(\mathcal{C}) \to \text{Obj}(\mathcal{C})$, que asignan a cada flecha $f$ su dominio $dom(f)$ y codominio $cod(f)$. Para una flecha $f$ con dominio $A$ y codominio $B$ escribiremos $f: A\to B$. Y para cada par de objetos $A,B$ definimos el \textcolor{red}{conjunto}
				$$\mathcal{C}(A,B) := \{f\in Mor(\mathcal{C}) \;|\; f:A\to B\}$$
				al que llamaremos \textit{Hom-set} y también escribiremos como $Hom_{\mathcal{C}}(A,B)$.
			\end{itemize}
			
		\end{definicion}
	\end{frame}
	
	\begin{frame}[t]
		\frametitle{¿Qué es una categoría?}
		\begin{definicion}[Categoría]
			\begin{itemize}
				\item Para cualquier terna de objetos $A,B,C$, la composición de morfismos,
				$$C_{A,B,C}: \mathcal{C}(A,B)\times\mathcal{C}(B,C)\to \mathcal{C}(A,C).$$
				Dados $f\in \mathcal{C}(A,B)$, $g\in \mathcal{C}(B,C)$, escribiremos $g\circ f$ para denotar $C_{A,B,C}(f,g)$.
				\[\begin{tikzcd}[ampersand replacement=\&,cramped]
					A \&\& B \&\& C
					\arrow["f", from=1-1, to=1-3]
					\arrow["{g\circ f}"', curve={height=24pt}, from=1-1, to=1-5]
					\arrow["g", from=1-3, to=1-5]
				\end{tikzcd}\]
				\item Para cada objeto $A$, una flecha identidad, $\textbf{1}_{A}: A\to A$.
			\end{itemize}
			
		\end{definicion}
	\end{frame}
	
	\begin{frame}[t]
		\frametitle{¿Qué es una categoría?}
		\begin{definicion}[Categoría]
			Tales que se satisfagan los siguientes axiomas:
			\begin{itemize}
				\item Asociatividad: para cualesquiera morfismos $f:A\to B$, $g:B\to C$, $h:C\to D$, se cumple que $$h\circ(g\circ f) = (h\circ g)\circ f$$
				\[\begin{tikzcd}[ampersand replacement=\&,cramped]
					A \&\&\& B \&\&\& C \&\&\& D
					\arrow["f", from=1-1, to=1-4]
					\arrow["{g\circ f}"', curve={height=24pt}, from=1-1, to=1-7]
					\arrow["g", from=1-4, to=1-7]
					\arrow["{h\circ g}", curve={height=-24pt}, from=1-4, to=1-10]
					\arrow["h", from=1-7, to=1-10]
				\end{tikzcd}\]
			\end{itemize}
		\end{definicion}
	\end{frame}
	
	
	\begin{frame}[t]
		\frametitle{¿Qué es una categoría?}
		\begin{definicion}[Categoría]
			\begin{itemize}
				\item Identidades: para cualquier morfismo $f:A\to B$ se cumple que $$f\circ \textbf{1}_{A} = f = \textbf{1}_{B}\circ f.$$
				\[\begin{tikzcd}[ampersand replacement=\&,cramped]
					A \&\&\& A \\
					\\
					\\
					\&\&\& B \&\&\& B
					\arrow["{1_{A}}", from=1-1, to=1-4]
					\arrow["{f\circ 1_{A}}"', from=1-1, to=4-4]
					\arrow["f", from=1-4, to=4-4]
					\arrow["{1_{B}\circ f}", from=1-4, to=4-7]
					\arrow["{1_{B}}"', from=4-4, to=4-7]
				\end{tikzcd}\]
			\end{itemize}
		$\hfill\blacktriangle$
		\end{definicion}
	\end{frame}
	
	\begin{frame}[t]
		\frametitle{Ejemplo: $\mathsf{Set}$}
			$\mathsf{Set}$ es la categoría cuyos objetos son conjuntos y sus morfismos son funciones entre conjuntos. En esta categoría
			\begin{itemize}
				\item \textbf{Flecha identidad:} Es la función identidad
				\begin{align*}
					 \mathbf{1}_{X}: X&\to X\\
					 x&\mapsto x
				\end{align*}
				\item \textbf{Composición:} Es la composición de funciones. Si $f:X\to Y$ y $g: Y\to Z$ son dos funciones, luego
				\begin{align*}
					g\circ f: X&\to Z\\
					x&\mapsto g(f(x))
				\end{align*}
				es su función composición.
			\end{itemize}
	\end{frame}
	
	
	\begin{frame}[t]
		\frametitle{Otros ejemplos}
		\begin{itemize}
			\item $\mathsf{Vect}$: Los objetos son espacios vectoriales y los morfismos son transformaciones lineales.
			\item $\mathsf{Pos}$: Los objetos son conjuntos parcialmente ordenados y los morfismos son funciones monótonas.
			\item $\mathsf{Top}$: Los objetos son espacios topológicos y los morfismos son los mapeos continuos.
			\item $\mathsf{Grp}$: Los objetos son grupos y morfismos son homomorfismos de grupos.
		\end{itemize}
	\end{frame}
	
	\begin{frame}[t]
		\frametitle{¿Qué es un functor?}
		\begin{definicion}[Functor]
			Un functor $F: \mathcal{C}\to\mathcal{D}$ está dado por:
			\begin{itemize}
				\item Un mapeo de objetos, que asigna un objeto $FA$ en $\mathcal{D}$ a cada objeto $A$ de $\mathcal{C}$.
				\item Un mapeo de flechas que asigna un morfismo $Ff:FA\to FB$ en $\mathcal{D}$ a cada morfismo $f:A\to B$ en $\mathcal{C}$ de tal forma que se preserven las identidades y composiciones, es decir: $F(g\circ f) = F(g)\circ F(f)$ y $F(1_{A}) = 1_{F(A)}$.	
			\end{itemize}
		$\hfill\blacktriangle$
		\end{definicion}
	\end{frame}
	
	\begin{frame}[t]
		\frametitle{Ejemplo}
			\begin{definicion}[Imagen directa]
				Sea $f:X\to Y$ una función y $A\subseteq X$ un subconjunto de $X$, la imagen directa de $X$ bajo $f$ es el conjunto $f[X]\subseteq Y$ que tiene a todos los elementos de $Y$ iguales a $f(x)$ para alguna $x\in A$:
				\begin{equation*}
					f[A] := \{ y \in Y \;|\; y = f(x) \text{ para } x\in A\}
				\end{equation*}
				$\hfill\blacktriangle$
			\end{definicion}
			
			\begin{block}{El functor potencia}
				En $\mathsf{Set}$ podemos definir el (endo)functor potencia $\mathcal{P}:\mathsf{Set}\to\mathsf{Set}$ cuyo mapeo de objetos envía un objeto $X$ a su conjunto potencia $\mathcal{P}X$ y el mapeo de morfismos envía una función $f:X\to Y$ a la imagen directa de $S\subseteq X$ bajo $f$, $\mathcal{P}f:\mathcal{P}X\to\mathcal{P}Y$.
			\end{block}
	\end{frame}
	
	\begin{frame}[t]
		\frametitle{¿Qué es una transformación natural?}
		\begin{definicion}[Transformación natural]
			Dados dos functores $S,T:\mathcal{C}\to\mathcal{D}$, una transformación natural $\alpha:S\Rightarrow T$ es un morfismo que asigna a cada objeto $c$ de $\mathcal{C}$ una flecha $\alpha_{c}: Sc\to Tc$ de $\mathcal{D}$ de tal forma que para cada flecha $f: c\to c'$ en $\mathcal{C}$ el siguiente diagrama 
			\[\begin{tikzcd}[ampersand replacement=\&]
				c \&\& Sc \&\&\& Tc \\
				\\
				\\
				{c'} \&\& {Sc'} \&\&\& {Tc'}
				\arrow["f"', from=1-1, to=4-1]
				\arrow["{\alpha_{c}}", from=1-3, to=1-6]
				\arrow["Sf"', from=1-3, to=4-3]
				\arrow["Tf", from=1-6, to=4-6]
				\arrow["{\alpha_{c'}}"', from=4-3, to=4-6]
			\end{tikzcd}\]
			es conmutativo.
			$\hfill\blacktriangle$
		\end{definicion}
	\end{frame}

	\begin{frame}[t]
		\frametitle{¿Qué es una transformación natural?}
		\begin{block}{Intuición}
			Si pensamos en los functores $S$ y $T$ como ``proyecciones'' de la ``imagen'' de una categoría sobre otra, entonces una transformación natural es un conjunto de flechas que de cierta forma ``traduce'' una imagen en otra.
		\end{block}
	\end{frame}

	\begin{frame}[t]
		\frametitle{¿Qué es una mónada?}
		\begin{block}{Broma 1.1}
			Una mónada es un burrito.
		\end{block}
		\begin{definicion}[Mónada]
			Una mónada $T = (T,\eta, \mu)$ en una categoría $\mathcal{C}$  consiste de un endofunctor $T: \mathcal{C}\to\mathcal{C}$ y dos transformaciones naturales $\eta: 1_{\mathcal{C}}\Rightarrow T$ y $\mu: T^{2}\Rightarrow T$ tales que los siguientes diagramas son conmutativos
			\[\begin{tikzcd}[ampersand replacement=\&]
				{T^{3}} \&\& {T^{2}} \&\& T \&\& {T^{2}} \&\& T \\
				\\
				{T^{2}} \&\& T \&\&\&\& T
				\arrow["{T\mu}", from=1-1, to=1-3]
				\arrow["{\mu T}"', from=1-1, to=3-1]
				\arrow["\mu", from=1-3, to=3-3]
				\arrow["{\eta T}", from=1-5, to=1-7]
				\arrow["1"', from=1-5, to=3-7]
				\arrow["\mu"{description}, from=1-7, to=3-7]
				\arrow["{T\eta}"', from=1-9, to=1-7]
				\arrow["1", from=1-9, to=3-7]
				\arrow["\mu"', from=3-1, to=3-3]
			\end{tikzcd}\]
		$\hfill\blacktriangle$
		\end{definicion}
	\end{frame}
	
	\begin{frame}[t]
		\frametitle{Ejemplo}
		Consideremos nuevamente el endofunctor potencia $\mathcal{P}: \mathsf{Set}\to\mathsf{Set}$. Sean $\eta_{Y}: Y\Rightarrow\mathcal{P}(Y)$ la operación de singleton 
		
		$$\eta_{Y}(y) = \{y\}$$
		
		y $\mu_{Y}:\mathcal{P}\mathcal{P}(Y)\Rightarrow\mathcal{P}(Y)$ la operación de unión 
		
		$$\mu_{Y}(\gamma) = \bigcup\gamma$$
		
		Como $\eta$ y $\mu$ son transformaciones naturales, entonces la terna $(\mathcal{P}, \{-\}, \bigcup)$ es una mónada en la categoría $\mathsf{Set}$.
	\end{frame}
	
	\begin{frame}[t]
		\frametitle{¿Qué es una categoría monoidal?}
		\begin{definicion}[Categoría monoidal]
			Una categoría monoidal $\mathbf{B} = \left\langle \mathbf{B}, \otimes, e, \alpha, \lambda, \rho\right\rangle$ es una categoría $\mathbf{B}$, un bifunctor $\otimes: \mathbf{B}\times\mathbf{B}\to\mathbf{B}$, un objeto $e\in\mathbf{B}$ y tres \textcolor{red}{transformaciones naturales} $\alpha, \lambda, \rho$. De forma explicita
			\begin{equation*}
				\alpha: a\otimes(b\otimes c)\cong(a\otimes b)\otimes c
			\end{equation*}
			
			\[\begin{tikzcd}[ampersand replacement=\&,cramped,column sep=small]
				{a\otimes(b\otimes(c\otimes d))} \&\& {(a\otimes b)\otimes(c\otimes d)} \&\& {((a\otimes b)\otimes c)\otimes d} \\
				\\
				{a\otimes((b\otimes c)\otimes d)} \&\&\&\& {(a\otimes(b\otimes c))\otimes d}
				\arrow["\alpha", from=1-1, to=1-3]
				\arrow["{1\otimes \alpha}"{description}, from=1-1, to=3-1]
				\arrow["\alpha", from=1-3, to=1-5]
				\arrow["\alpha"', from=3-1, to=3-5]
				\arrow["{\alpha\otimes 1}"{description}, from=3-5, to=1-5]
			\end{tikzcd}\]
		\end{definicion}
	\end{frame}
	
	\nocite{*}
	\begin{frame}[t, allowframebreaks]
		\frametitle{Referencias}
		\bibliographystyle{abbrv}
		\bibliography{references}
	\end{frame}
\end{document}
